\anonsection{Введение}

Введение дипломной работы четко структурировано (незначительно отличается в разных учебных заведениях) и включает блоки:

    Формулировка темы;
    Цель дипломной работы;
    Задачи;
    Методы исследования;
    Объект исследования;
    Предмет;
    Гипотеза;
    Степень изученности вопроса;
    Новизна темы;
    Актуальность.

Уточним: цель исследования — зачем оно пишется, задачи – через решение каких вопросов цель достигается.
Когда студенты пишут введение самостоятельно, то часто путают предмет и объект. Правильно считать объект главным, на что обращено внимание исследователя, а предмет – изучаемой частью, областью объекта.
Рекомендуем написать введение после завершения текста ВКР: в процессе изучения зачастую происходят изменения, корректировки первоначального плана, задач, методов и целей, объекта, предмета исследования.

Выпускная квалификационная работа (ВКР) бакалавра или магистра – научное исследование, в котором резюмируется учебная деятельность выпускника, и определяется уровень его подготовленности к работе по профилю специальности. При всех нюансах защит в различных вузах — где-то требуется презентация, а где-то – чертежи — наличие пояснительной записки обязательно. А она начинается с вступительной части. Как написать введение к диплому, чтобы читая его, у рецензентов или членов Государственной экзаменационной комиссии (ГЭК)  сложилось благоприятное впечатление о дипломнике и проекте?

Структура введения дипломной работы обязательно должна быть четкой. Читать введение в форме «портянки» — текста без абзацев и списков – неудобно и тягостно. Рецензент может сделать замечания, которые негативно повлияют на оценку. Члены ГЭК, в свою очередь, составят для себя невысокое мнение о способностях защищающегося. И захотят опровергнуть (или подтвердить) его соответствующими вопросами. При этом не факт, что на них можно будет правильно ответить: помешает эмоциональная возбуждённость дипломника, нервы, а иногда – и растерянность перед аудиторией. Не зря студентам советуют чаще выступать перед сокурсниками, преподавателями и просто слушателями.

Введение начинают с уточнения его объёма. Полезно посоветоваться с руководителем дипломного проекта или работы, а также  с выпускниками того же вуза прошлых лет. Обычно введение не занимает более 10% от объёма расчётно-пояснительной записки.

Пишется введение тогда, когда уже согласованы разделы дипломной работы, подготовлена «рыба» презентации и подписаны чертежи. Введение к диплому должно быть написано в соответствии с действующими стандартами. Поскольку дипломная работа или проект являются, по сути, развёрнутым отчётом по студенческому исследованию, то к расчётно-пояснительным запискам применяются требования ГОСТ  7.32-2001 «Отчёт о научно-исследовательской работе. Структура и правила оформления». Поэтому, перед окончательной сборкой записки, полезно ознакомиться с данным документом.

Далее рассмотрим структуру введения к дипломной работе.
Актуальность исследования

Вторичность работы – факт, встречающийся часто, хотя он мало зависит от автора, написавшего диплом. Всё определяется темой ВКР, которая утверждается на профилирующей кафедре с учётом специфики базы практики или места будущей работы выпускника. Обоснованием для актуальности считаются:

    Существующие проблемы в отрасли.
    Уровень известных исследований по теме.
    Практические сложности в дипломной работе, с которыми столкнулся автор.
    Результаты собственного анализа ситуации по теме.

Актуальность исследований должна быть конкретной. Если это диплом на техническую тему – то это цифры, характеризующие фактическую трудоёмкость существующих технологий или несовершенство оборудования. Диплом гуманитарной тематики должен содержать  социометрические показатели объекта исследований.

Актуальность формулируется по конкретным пунктам. Чем их больше, тем более своевременна тема, на которую подготовлен диплом или ВКР.

    Пример актуальности темы дипломной работы:

    Настоящая работа посвящена рассмотрению проблем связанных с правовым регулированием заключения государственного контракта в РФ. Актуальность темы исследования определяется значимостью ГМК, который выполняет задачи по обеспечению государственных нужд, реализации целевых программ и формированию резервного фонда, обеспечивает экономико-правовое и социально-политическое развитие государства. Государственные закупки имеют важное политическое и экономическое значение и являются составляющей частью обеспечительной функции государства. Законодательство о публичных закупках постоянно развивается, о чем свидетельствует создание новой системы посредством принятия Закона о контрактной системе.

    Таким образом, актуальность темы исследования определяется политической и экономической значимостью государственных закупок и необходимостью анализа его правового регулирования в связи с изменениями законодательства.

    Актуальность темы исследования также подтверждается и вниманием к данному институту со стороны исследователей в области экономики и права.

Объект исследования

От общего к частному, или дедукция  — метод, который считается основным при ответе на вопрос, как написать введение.  При этом автор ВКР постепенно переходит к одному из «слабых звеньев», которое тормозит технический или интеллектуальный прогресс. Фактически объект исследований на 80…90% совпадает с темой, которой посвящён диплом.

Описание объекта может включать в себя:

    Общую характеристику (по пунктам, и — желательно – с количественными оценками измерений; они обычно выполняются в период преддипломной практики).
    Структуру объекта.
    Описание взаимосвязей отдельных частей объекта.
    Логические выводы о необходимости модернизации или изменения объекта.

Наличие рисунков или схем здесь обязательно, причём их стоит повторить не только в записке к ВКР, но и в визуально-графических материалах – чертежах, плакатах, слайдах презентации. С повышением степени наглядности этих материалов оценка, которую получит диплом, гарантированно возрастёт.

    Пример выделения объекта исследования:

        Объектом исследования являются общественные отношения в сфере заключения ГМК в РФ.
        Объект исследования — ООО «Лента».
        Объект исследования в дипломной работе: психическая адаптация к воинской службе

Предмет исследования

Следуя дедуктивному методу, опускаемся на следующую ступень, и детализируем суть работ, которые содержит диплом. Ведь за несколько месяцев дипломного проектирования сколько-нибудь сложный объект исследования изучить и проанализировать невозможно. Поэтому, совместно с научным руководителем, выбирается самая значимая часть объекта исследований, к которой в дальнейшем будут применяться все методы, программное обеспечение и т.д.

При характеристике предмета исследования – а это может быть технологический процесс, финансовая структура, социологическая ячейка и пр. – обязательно отмечается его сложность, трудности и выявленные ограничения существующих методик. Без этого проект (а ВКР – в особенности) покажется членам ГЭК легковесным. В результате возникнет множество вопросов, которых можно было бы и избежать. Исключение составляют так называемые «спровоцированные» вопросы,  ответы на которые автору хорошо известны.

    Пример предмета исследования в дипломной работе:

        Предмет исследования – рентабельность деятельности ООО «Лента»
        Предмет исследования – эффективность управления денежными потоками в ООО «Мануфактура Балина»
        Предметом исследования являются нормы действующего законодательства и теоретические представления исследователей в сфере понятия, формирования, исполнения ГМК.

Цель работы и задачи исследования

Цель и задачи исследования — обязательная составляющая введения дипломной работы.

Цель работы связывается с одной стороны – с предметом исследования, а, с другой – со сложностями, преодолеваемыми в случае применения полученной информации. Цель формулируется развёрнуто, и должна содержать хотя бы три позиции. Первая из них заключается в теоретических новинках, которые малоизвестны, вторая касается выбранной методологии научных исследований, а третья посвящается перспективам дальнейших разработок  (это  обязательно для ВКР на квалификационный уровень магистра).

Задачи исследования обозначаются так, чтобы каждая из них относилась к решению определённой цели. Логичнее перечислять задачи последовательно, от простой к более сложным, причём хорошо, когда решение предыдущей задачи является основанием для работы над следующей.

Задачи исследования должны опираться на математический или статистический аппарат, который соответствует общей методологии, описанной ранее.

    Пример цели и задач исследования:

    Цель работы — анализ перспектив повышения эффективности управления денежными потоками предприятия, на примере ООО «Мануфактура Балина».
    В соответствии с этой целью в дипломной работе поставлены следующие задачи:
    —   рассмотреть теоретические и методологические подходы к анализу денежных потоков предприятия;
    —   проанализировать эффективность системы управления денежными потоками в ООО «Мануфактура Балина»;
    —   разработать меры по оптимизации управления денежными потоками ООО «Мануфактура Балина».

    Целью настоящего исследования является анализ правового регулирования заключения ГМК для выявления проблем в данной сфере и формулирования предложений по их устранению.
    Для достижения поставленной цели необходимо решение следующих задач:
    — провести анализ понятия ГМК;
    — определить виды ГМК;
    — рассмотреть правовое регулирование ГМП;
    — определить правовой статус сторон по ГМК;
    — рассмотреть процесс планирования и обязательного общественного обсуждения закупок до заключения ГМК;
    — провести анализ правового регулирования и организации контроля в сфере заключения ГМК.

Возможно, вам будет интересно: Как написать хороший реферат.
Гипотезы и теории

Проект, даже рядовой, не обходится без научной базы. При этом способы решения конкретных задач должны соответствовать возможностям выпускника. Например, применение  сложного программного комплекса для решения сравнительно простой проблемы  может вызвать у членов ГЭК и рецензента ненужные подозрения, не говоря уже о нецелесообразности «стрельбы из пушек по воробьям».

Более  высокий уровень сложности гипотез допустим тогда, когда автору светит поступление в аспирантуру. Но и здесь следует быть осторожным. Существующие программы выявления текстового плагиата (а их может использовать рецензент) могут поставить крест на амбициях, которые лелеет автор,  как ему кажется, высоконаучной выпускной работы.

Ещё одна опасность возникает тогда, когда магистрант вторгается на запретную территорию – использует гипотезы и/или теории от другой научной школы, которая конкурирует с той, что принята  в родной alma mater. Это характерно для проектов особо амбициозных исполнителей.

    Пример гипотезы исследования:

    Гипотеза исследования. Мы предполагаем, что профилактика девиантного поведения младших школьников будет эффективной, если:
    1) будут учитываться возрастные и психологические особенности данного возраста;
    2) программа будет нацелена на формировании положительных личностных качеств и поведенческих реакций, а также формирование потребности в социальном признании и самоутверждении.

Возможно, вам будет интересно: Пример заполнения дневника по практике.
Методы исследования

Пройдя непростой этап гипотез и теоретических предпосылок,  в дальнейшем  введение уже увереннее движется к финишу. Здесь введение содержит в себе перечень конкретных методов исследования, которые привели к нужному результату. При этом стоит разграничить методы, применяемые в дипломной работе по блоку гуманитарных дисциплин, с методами, характерными для проектов технической тематики. В первом случае преобладают статистические оценки, во втором — программы математического планирования эксперимента.

И в первом, и во втором случае применяется софт из того, что традиционно используется на выпускающей кафедре. Не потому, что они лучше, а потому, что у научного руководителя наработан большой опыт, дающий нужный результат. Материалы могут содержать и негативную оценку исследований, но на практике такой вариант встречается редко.

    Например, методы исследования:
    – теоретические методы исследования: анализ психолого-педагогической литературы по проблеме исследования,
    – эмпирические методы исследования: анкетирование, тестирование, эксперимент, методы математической статистики.

Возможно, вам будет интересно: Как написать отчет по практике самостоятельно.
Научная новизна и практическая ценность

Диплом, не содержащий научной новизны, практических выводов и рекомендаций, высокой оценки от ГЭК не получит. Пункты, касающиеся научной новизны, должны базироваться на полученных результатах, которые ещё неизвестны в данной отрасли. Подтверждением этому служат:

    Публикации в специализированных научных сборниках (для ВКР это должны быть категорированные сборники, распространяемые минимум в СНГ).
    Доклады на региональных, федеральных или международных студенческих конференциях.
    Работы, занявшие призовые места на завершающих этапах студенческих олимпиад.
    Патенты (хотя бы декларативные), или, как исключение, положительные решения по ним.

Ссылки приводятся в приложениях к пояснительной записке, и в списке использованной литературы.

В качестве элементов практической ценности выступают:

    Акты внедрения результатов исследований.
    Акты положительных испытаний разработанного устройства или методики.
    Благоприятные отзывы руководителей предприятий и организаций, где применены  основные положения проекта.

    Пример:

    Практическая значимость исследования, проведенного в дипломной работе состоит в выводах и предложениях по результатам проведенного анализа рентабельности ООО «Лента». По его результатам разработаны меры по повышению рентабельности ООО «Лента». Разработанные рекомендации являются, в соответствии с проведенными расчетами, экономически обоснованными и способны повысить рентабельность исследуемого предприятия.

    Практическая значимость исследования, проведенного в работе состоит в выводах и предложениях по результатам проведенного анализа денежных потоков ООО «Мануфактура Балина». По его результатам разработаны меры по повышению эффективности управления денежными потоками исследуемого предприятия. Выводы и рекомендации, предложенные в работе, являются в соответствии с проведенными расчетами экономически обоснованными и способны оптимизировать управление денежными потоками в ООО «Мануфактура Балина».

Введение является мостиком к конкретной  реализации всех положений, перечисленных выше.

\clearpage