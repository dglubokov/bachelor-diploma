\anonsection{Введение}

Основная цель преддипломной практики --
теоретический анализ существующих методов
для обнаружения и прогнозирования ошибок на устройствах.
Данный анализ необходим для решения поставленной
задачи в дипломной работе, которая заключается
в прогнозировании неисправностей в станках лазерной резки
Навигатор КС-12В фирмы ВНИТЭП.

Для решения обозначенной проблемы
необходимо провести анализ предметной области,
а именно предоставить
функциональное и модульное описание станка Навигатор.
Кроме этого, необходимо дать мотивировка проблемы
автоматического обнаружения неисправности на станках данной модели.
Также, в работе необходимо провести анализ существующих
актуальный решений для похожих задач прогнозирования и обнаружения неисправностей.
Для каждого решения необходимо привести критику в виде выявленных преимуществ и недостатков
каждого метода и подхода.

Еще одной задачей преддипломной практики является
определение места разрабатываемого решения в существующую
систему компании Omnicube, с последующим коротким ее описанием.

В процессе прохождения преддипломной практики необходимо освоить следующие компетенции:

- ОК-7 Способностью к самоорганизации и самообразованию;

- ОПК-5 Способностью решать стандартные задачи профессиональной деятельности на основе информационной и библиографической культуры с применением информационно - коммуникационных технологий и с учетом основных требований информационной безопасности.

- ПК-3 Способностью обосновывать принимаемые проектные решения, осуществлять постановку и выполнять эксперименты по проверке их корректности и эффективности.

\clearpage