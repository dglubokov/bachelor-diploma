\anonsection{Введение}

Основная цель данной дипломной работы --
разработка модуля для прогнозирования
и обнаружения ошибок и сбоев на станках лазерной резки Навигатор КС-12В компании ВНИТЭП.
Проблема прогнозирования неисправностей актуальна не только для станков данной модели,
но и для других разнообразных видов производственных устройств,
так как не существует общепринятых стандартов для задач подобного рода,
а сами задачи решаются экспериментально на основе последний достижений науки.

Мотивацией для создания системы прогнозирования и выявления неисправностей
является сохранение ресурсов компании Сеспель, которая использует станки Навигатор,
а именно сокращение времени и денежных средств по диагностике и устранению неисправностей.

Для определения источников и типов неисправностей
в дипломной работе было дано функциональное и модульное описание станка Навигатор КС-12В.
На основе этого описания определялись состояния ошибок и сбоев,
а также на основе рекомендаций инженеров компании ВНИТЭП.

Решение поставленной проблемы потребовало определения лучших методов и подходов.
Для этого в дипломной работе были исследованы и проанализированы существующие
решения, направленные на предсказания и выявление неисправностей
устройств.
Все решения, описанные в работе, строятся на основе принципов
интеллектуального анализа данных и анализа временных рядов.
Эти две области также были описаны в работе.

В процессе анализа предметной области
были выявлены задачи, которые необходимо было решить для достижения поставленной цели.
Все задачи были перечислены.

В процессе анализа были выявлены достоинства и недостатки существующих решений,
а также сделан выбор инструментов и методов на основе анализа,
а именно набор библиотек для работы с данными с использованием языка Python
и методы для прогнозирования и обнаружения сбоев.

Для прогнозирования временных рядов данных со станка
была использована нейронная сеть LSTM.
В работе был дан анализ архитектуры данной нейронной сети,
а также описаны дополнительные методы
для оптимизации результатов работы этой нейроной сети.

Для выявления ошибок и сбоев на станке
используются предсказанные значения,
а также общие тренды параметров лазера: температуры и мощности.
Основные критерии неисправностей -- выход за крайние значения
температуры и мощности лазера в течение продолжительного времени.

В целях выявления паттернов для состояний станка,
был использован алгоритм k-Shape,
который направлен на кластеризацию многомерных временных рядов.
Данный алгоритм является лучшим для кластеризации временных рядов.
В работе описаны принципы алгоритма в сравнении с другими алгоритмами.
Также, стоит отметить, что выявленные паттерны используются алгоритмом градиентного бустинга,
который также описан в работе.

Разработанный модуль является частью программного комплекса компании Omnicube.
В работе было дано описание этой системы,
а также принципы интеграции созданного модуля в систему.

\clearpage