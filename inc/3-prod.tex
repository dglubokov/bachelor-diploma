\section{Внедрение в эксплутацию и результаты}

\subsection{Предобработка и первичный анализ данных}

(диаграмма последовательности выполненных задач)

(Препроцессинг и первичный анализ тестовая)

(Препроцессинг и первичный анализ основная)


\subsection{Архитектура разработанного решения}

(диаграмма пайплайна)


\subsection{Обучение LSTM}

(Оптимальное историческое окно обучения)

(Адаптивное окно предсказания и оптимальные окна обучения)

(диаграмма с окнами)

Обучение моделей на временных рядах может происходить различными способами.
Основным способ является обучение на основе скользящего окна.
Предположим, что есть временной ряд $TS = \{x_1, ..., x_n\}$.
Тогда, скользящим окном называется фрагмент ряда $TS$ размера $m$ с шагом $s$.
Тем самым, исходный ряд разбивается на сегменты одинакового размера: $TS = \{S_1, ..., S_{\frac{n}{s}}\}$.

(LSTM описание)

(кросс-валидация, оценка и прочее)

\subsection{Кластеризация на основе k-Shape}

(оптимальное число кластеров)

(кластеризация)

(описание паттернов)

\subsection{Градиентный бустинг}

(обучение на основе кластеров)

(кросс-валидация)

(тестирование и оценка)

\subsection{Статические критерии}

(описание самостоятельных критериев на основе трендов)

\subsection{Введение в эксплуатацию}

(система оповещения)

(интеграция с omniprocessing и соединение моделей)

(Kuberntetes развертывание)



\clearpage