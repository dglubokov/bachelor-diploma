\anonsection{Заключение}

Отвечая на вопрос «Как правильно написать заключение?», отметим, что оно строится по следующему плану:
•    Вводная часть;
•    Достигнута ли цель исследования;
•    Каким образом решились основные задачи дипломной работы;
•    Выводы по теоретической, практической части;
•    Практическое применение результатов исследования;
•    Перспективы дальнейшей разработки темы.

Заключение дипломной работы составляет 4-5 страниц. Здесь необходимо написать собственные мысли, суждения, указать  выводы, описать результаты работы.
\\
Дипломная работа – это огромный массив информации, который студент должен исследовать и правильно структурировать. Но вот, когда необходимые разделы готовы, перед учащимися стоит проблема того, как написать заключение к дипломной работе.

Заключение у многих вызывает сложности. Правильные выводы должны состоять не только из описания процессов и исследований, еще нужно подвести итог результатам экспериментов, анализов, логично сформулировать окончание дипломной работы.
Попытаемся разобраться, как правильно написать заключение дипломной работы.

Вам может быть интересно: Как написать дипломную работу самостоятельно. Образец
Компоненты в подведении итогов дипломного проекта

Подведение итогов является обязательным логическим окончанием исследования, в рамках проведения работы. Выводы ВКР должны размещаться в конце исследования, после основной части включающей теоретическую и практическую информацию по теме, но перед списком используемых источников работы.

Структурные элементы в написании заключения:
—    тезисы, которые выделил студент во время изучения теоретического пласта информации, содержащие предложения о возможности практического применения полученной информации;
—    короткое описание, оценка проведенных практических экспериментов и проектов;
—    обоснование значимости и отличия темы диплома, от остальных примеров дипломных проектов;
—    предложения практического характера, сгенерированные во время исследования;
—    можно написать предложения, касающиеся перспективы развития тематики, возможные темы схожих исследований дипломных работ;
—    выводы.

Заключительная часть ВКР — подведение общих итогов, где подтверждается или опровергается предложенная ранее гипотеза, автор отвечает на вопросы, поставленные перед собой в начале исследования, обосновывает, достиг он поставленных целей во время проведения экспериментов или нет.

    Подводя итог, можно сказать, основная функция заключения ВКР – это демонстрация выводов, к которым пришел студент после исследования темы дипломной работы, анализ и собственные предложения.

Стоит акцентировать внимание, писать заключение, как и введение, следует по определенной структуре, шаблону. Главным отличием является то, что введение имеет конкретные разделы, а подведение итогов должно основываться исключительно на ваших собственных мыслях.

Читайте также: Как правильно написать введение к дипломной работе
Объем и структура заключения в дипломной работе

Заключение дипломной работы в объеме обычно меньше вводной части, но если они будут содержать одинаковое  количество страниц, это не станет ошибкой. Писать в заключении необходимо максимальное количество значимых фактов, только лаконичную и структурированную информацию. Средний объем заключения к диплому три-четыре страницы.

Примерная структура заключения

Вводная часть. Не нужно начинать писать заключение с результатов своего исследования. В первых строках правильнее писать о важности изучения проблемы и актуальности исследования.

Основная часть. Содержит краткие итоги, к которым пришли во время изучения исследуемого вопроса, подтверждение или опровержение ранее сформулированной гипотезы. Выводы по теоретической и практической частям должны приводиться равномерно.  Ключевые мысли должны не растекаться по тексту, а быть четко структурированы. Сделать выводы максимально правильно помогут задачи, поставленные в начале исследования. Отдельно необходимо отметить достигнуты ли поставленные цели во время проведения исследования проблемы, и смогли ли вы ответить на все вопросы. Закончить можно выводом: подтверждается гипотеза или опровергается.

Заключительная часть. Здесь нужно привести обоснование практической ценности использования полученной вами информации. Плюсом станут сформулированные советы по поводу решения проблемы, предложения новых подходов, определение возможностей использования полученной информации в практичных целях.

Советуем прочитать: Как написать хороший реферат
Лайфхаки в написании окончания дипломной работы

➢    Чтобы быстрее сделать заключительную часть, нужно собрать все выводы, которые сделали в дипломной работе после каждого раздела или части и, видоизменив,  представить их как вариант выводов. Способ не сможет принести отличной оценки, но он имеет право на жизнь.
Претендуя на высокую оценку, выводы нужно тщательно обработать. Подводя итоги исследования, не стоит писать выводы отдельно для  практической и теоретической части, лучше сделать это комплексно, связав их между собой.

➢    Вы не специализируетесь в написании научных проектов, и научный стиль написания трудно дается? Используйте шаблонные фразы! К таким фразам можно отнести:
Мы приходим к выводу о…
В заключении отметим, что…
В нашем исследовании мы выяснили…

➢    Если с написанием заключения возникли трудности, в качестве образца возьмите пример заключения дипломной работы другого студента. Лучше взять образец, написанный под руководством вашего куратора. Имея перед собой несколько образцов, несложно найти правильные слова, придерживаться определенной структуры и следовать определенному плану написания.

Написание заключительной главы ВКР  —  один из главных процессов создания научного проекта. Преподаватели, оценивая дипломное исследование, обращают внимание именно на введение и заключение проекта, именно там находится основная информация.

Цельное, структурированное заключение будет отличным окончанием диплома и станет основой для написания речи для защиты дипломной работы.

Пример заключения дипломной работы


\clearpage