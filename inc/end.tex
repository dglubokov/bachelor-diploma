\anonsection{Заключение}

Результатом данной дипломной работы является решение проблемы
автоматического прогнозирования и выявления неисправностей
на станках лазерной резки Навигатор КС-12В компании ВНИТЭП.
Для получения результата был решен ряд задач.

Была обозначена проблема выявления ошибок и сбоев на станках лазерной резки модели Навигатор КС-12В,
а также в устройствах вообще.
Описаны особенности данной проблемы, а также ее теоретические аспекты. Проведен анализ
существующих решений, базирующихся на различных методах, подходах и идеях.
Для каждого случая была дана критика, а именно обозначены достоинства и недостатки каждого решения.
Также были обозначены задачи, решения которых может помочь достичь решения проблемы выявления ошибок и прогнозирования.

Проведен анализ системы компании Omnicube, в которую
будет интегрировано решение для прогнозирования и обнаружения ошибок станков Навигатор.

На основе предобработки данных со станка лазерной резки Навигатор КС-12В
был проведен первичный анализ особенностей и свойств выделенных параметров станка:
температуры и мощности лазера.
Для этого был проведен ряд тестов: на нормальность и на стационарность рядов.


Для задач прогнозирования была выбрана рекуррентная нейронная сеть LSTM.
Описаны преимущества перед другими архитектурами нейронных сетей
для задач прогнозирования временных рядов.
Было дано подробное описание работы сети LSTM.
Также был описан принцип, по которому определяется оптимальное окно обучения.
Выявлено оптимальное окно для тренировки модели предсказания.
Описано использование модели предсказания -- нейронной сети LSTM.
Дана характеристика гиперпараметров модели.


Кроме этого, был проведен анализ алгоритмов кластеризации,
а также выбран и описан алгоритм k-Shape для кластеризации многомерных временных рядов.
Описано использование алгоритма k-Shape для кластеризации,
а также результаты этого использования.
Выявлены паттерны состояний станка, который могут
в дальнейшем будут использоваться алгоритмом градиентного бустинга.
Обозначены проблемы оптимизации данного алгоритма.

Дано краткое описание используемых для разработки инструментов,
а именно анализ библиотек и фреймворков для задач анализа данных и задач машинного обучения,
предназначенных для языка Python.

В конце дано описание принципов интеграции с системой компании Omnicube.

В заключении можно сказать, что все поставленные задачи выполнены,
а цель дипломной работы достигнута.

\clearpage